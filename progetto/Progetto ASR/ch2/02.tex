\chapter{Speech Features \& Embeddings}
\label{ch:features}
In molte applicazioni di speech processing è spesso conveniente ed utile avere una rappresentazione parametrica o alternativa dell'informazione
trasportata dal segnale. L'informazione trasportata dal segnale infatti include importanti fattori come il pitch period, il modello dell'impulso glottale,
l'eccitazione, il modello del tratto vocale: tutte informazioni utili a poter distinguere i diversi speaker in task come lo speaker identification
e lo speaker verification. L'obiettivo della Digital Speech Processing (DSP) è proprio quello di trovare metodi per poter elaborare i segnali speech audio
e poter quindi estrarre queste particolari informazioni in forma parametrica, avendo difatti una \textbf{rappresentazione alternativa} del segnale. 

Questo permette difatti di andare ad estrarre \textbf{speech features}, dunque parametri o caratteristiche, che possiamo dare in input a modelli di machine learning,
per permette alle reti di poter eseguire particolari combiti. Spesso però il machine learning ha una duplice valenza: è spesso usato sia come "modello" di elaborazione
delle features, sia come modello per estrarre rappresentazioni più raffinate delle features in input. Queste rappresentazioni più raffinate prendono il nome
di \textbf{speech embeddings}, e permettono di aggiungere un ulteriore informazione: permette difatti di aggregare i contributi di ogni singola features in ingresso,
cercando di ottenere rappresentazioni non lineari più complesse e talvolte più discriminanti.

\section{Speech Features}
In ogni caso, le proprietà dello speech cambiano lentamente nel tempo e nella frequenza. Per poter fronteggiare questo fenomeno si ricorre spesso a
metodi di analisi definiti come \textit{short time}, dove andiamo ad eseguire una analisi frame per frame. Questo deriva dal fatto che effettuando una analisi
short time a livello di frame, andiamo a processare singoli segmenti cui possiamo assumere abbiano proprietà fisse e permettono di estrarre le informazioni relative
per lo speech processing. Le metodologie di short time possono essere o basate su tempo o su frequenza. L'analisi time-domain è utile per andare ad isolare
segmenti dove è presente un pezzo di segnale \textit{voiced} oppure \textit{unvoiced}. Permette infatti di isoalre meglio segmenti dal rumore di fondo, utili per massimizzare
ulteriormente l'estrazione di features solo su segmenti definiti come voiced, ad esempio integrando un \textbf{Voice Activity Detector (VAD)}, che tronca andando a valutare
l'energia del segmento, ad esempio con analisi \textbf{Short Time Energy}. 

Tuttavia, nei task di \textbf{Speaker Verification} e \textbf{Speaker Identification}, conviene trattare di metodi short time basati sulla frequenza,
dal momento che molte informazioni riguardanti formanti e armoniche (utili per poter modellare un filtro che rappresenti il tratto vocale di una persona, quindi un modello
che riesce a parametrizzare), oltre che informazioni definibili come "percettive". Alcune features basate su metodi frequency-domain soon gli MFCCs (Mel Frequency Cepstral Coefficients).
Talvolta abbianiamo a queste features anche altri valori di derivata prima o seconda, per poter esprimere anche la variabilità dell'analisi short time, come i metodi Delta e Delta Delta.

\subsection{Cepstrum}
Il cepstrum è definito come la IDFT della log-magnitudo della DFT del segnale, ovvero:
\[ c[n] = F^{-1}[log(|F[x[n]]|) ]\]
Dove i due operatori sono la trasformata discreta di fourier DFT e la inverse DFT, la IDFT. A partire del cepstrum possiamo definire una rappresentazione alternativa,
ma non ci basta ancora: vorremmo smussare le attenuazioni, estraendo anche componenti come il pitch e sopratutto modellare l'inviluppo della trasformata stessa.
Tuttavia il cepstrum ci permette di poter separare la sorgente (voiced o treno di impulsi, e unvoiced o rumore gaussiano) ed il filtro (il tratto vocale), riuscendo quindi 
a modelleare in maniera compatta l'inviluppo spettrale. 

\subsection{Filter Banks e Mel-Scale}
Lo smooth con filterbanks riesce ad approssimare piccole variazioni, rendendo le features più robuste e meno rumorose. I filterbanks sono una serie di filtri
triangolari che permettono di andare a catturare selettivamente energia da uno specifico range di frequenze, pesando e sommando i bins spettrali in quella regione.
Questo processo permette di approssimare lo spettro e fornire una rappresentazione compatta di come l'energia è distribuita lungo le diverse bande di frequenza.
La rappresentazione smoothed cattura una forma complessiva dello spettro definita come "envelope", esaltando l'importanza delle features da un punto di vista percettivo.

\subsection{Mel-Frequency Cepstral Coefficients (MFCCs) }
La scala di Mel è una scala percettiva di pitch che approssima il modo in cui gli umani percepiscono i suoni delle frequenze. La scala di Mel è progettata per riflettere una relazione
non lineare tra la frequenza fisica e il tono percepito, risultando più sensibile alle basse frequenze, il quale riesce a farci a capire come l'orecchio umano percepisce
il suono. Formiamo quindi un filterbank cosicché i centri dei triangoli sono le frequenze che corrispondono a equal distanza sulla scala di mel. Effettuiamo quindi
una analisi short time dove applichiamo alla trasformata DFT i valori pesi del filtro filteranks:
\[ MF_m[r] = \frac{1}{A_r} \sum_{l=L_r}^{U_r} |V_r[k]X_m[k]|^2\]
Dove $V_r[k]$ è la funzione di peso per il filtro r-esimo filtrando gli indici della DFT da $L_r$ e $U_r$ e $A_r$ definito come fattore di normalizzazione per il filtro r-esimo di mel. 
\[ A_r =\sum_{l=L_r}^{U_r} |V_r[k]|^2\]
L'inviluppo di mel chiaramente modella le basse frequenze accuratamente, il quale è molto importante perché è qua che risiedono tutte le formanti. 
Le alte frequenze sono poco modellate, il quale generalmente è dove non è presente la maggior parte dell'energia. 

Tuttavia, se cambiamo il come calcolare la trasformata (cambiamo operatore), usando la Discrete Cosine Transform (DCT), utile per decorrelare dati sequenzialmente correlati,
prendiamo la DCT del Log-Mel Spectrum, il quale è conosciuto come \textbf{Mel-Frequency Cepstral Coefficient (MFCC)}. Eseguiamo prima una mappatura nella frequenza di mel,
poi prendiamo il log spettro e infine applichiamo la IDCT. Ad ogni frame m-esimo, la trasformata del log della magnitudo del risultato dei filtri viene calcolato per 
formare la funzione \textbf{$MFCC_m[n]$}, definita come:
\[ MFCC_m[n] = \frac{1}{R}\sum_{r=1}^{R}log(MF_m[r])cos\left[ \frac{2\pi}{R} \left( r + \frac{1}{2} \right) \right]\]

\subsection{Delta e Delta Delta}
Il segnale acustico è una sequenza di trasizioni tra fonemi, quindi dobbiamo capire quando riconoscerne uno e scinderlo dall'altro. Eseguendo una analisi short time ci portiamo
appresso questo fatto, e bisogna tenerne in considerazione quando calcoliamo le features. Tuttavia spesso è più informativo analizzare la forma generale della traccia
della features in relazione con le variazioni acustiche. Un metodo comune per estrarre informazioni circa le transizioni per determinare la prima differenza delle 
feature, conosciute come delta di feature. Specificamente, per una feature $f_k$ al tempo k, il corrispondente delta è definito come:
\[ \Delta _k = f_k - f_{k-1}\]
Mentre la delta-delta è definita come:
\[ \Delta \Delta_k = \Delta_k - \Delta_{k-1} \]
Queste ulteriori features permettono di definire  ed interpretrare il segnale come la derivata prima e la derivata seconda. I delta e i deltas sono componenti 
classici degli algoritmi di machine learning.

\section{Speech Embeddings}
L'obiettivo degli speech embeddings è quello di trasforare le sequenze di features estratte a partire dei metodi descritti prima
in un vettore che cattura le caratteristiche distintive di ogni singolo speaker, rendendolo adatto per compiti come la speaker verification ed identification. 
Questo vettore processato a partire dalle speech features prende il nome di \textbf{embeddings}. In letteratura sono stati usati diversi
embeddings, spesso combinando diverse features tra di loro, sebbene oggi la strada più promettente è quella dell'utilizzo di metodi di deep learning. 
Gli embeddings estratti dai metodi di deep learning rappresentano l'identità di una persona tramite un vettore a dimensione fissa codificato da una espressione
vocale di lunghezza variabile. 

\subsection{I-Vectors e Deep Embeddings}
Gli i-vectors sono stati per anni una rappresentazione standard e di successo per la speaker recognition.
questi sono vettori a bassa dimensione derivati da un modello statistico chiamato Total Variability Space. Essi catturano 
sia la variabilità legata al parlante che quella legata al canale di comunicazione. Nonostante la loro comprovata efficacia, 
mostrano alcune limitazioni, soprattutto quando la durata dell'espressione vocale è breve. 
L'introduzione delle Deep Neural Networks ha rivoluzionato il campo degli embeddings: l'idea è di addestrare una DNN per produrre 
embedding che siano altamente discriminativi per l'identità del parlante. 
Un approccio chiave è l'estrazione di embedding da segmenti di parlato. Invece di aggregare informazioni su un'intera espressione 
vocale all'inizio del processo, le DNN possono elaborare finestre temporali più piccole del segnale (segmenti) e poi aggregare le 
informazioni a un livello successivo. Ad esempio, gli autori di \cite{snyder2017deep} hanno proposto un sistema in cui gli embeddings 
vengono estratti da una rete di deep learning. La caratteristica distintiva di questo approccio è l'utilizzo di pooling temporale, che aggrega le informazioni 
estratte dalla rete su segmenti di input. Questo permettealla rete di essere addestrata per discriminare gli speakers direttamente sulla base 
del segmento vocale. 

\subsection{X-Vectors e TDNNs}
L'evoluzione degli embedding basati su DNN ha portato allo sviluppo degli x-vectors, che rappresentano una delle tecniche più avanzate per la 
speaker recognition. Come descritto da \cite{snyder2018x}, gli x-vectors sono embedding DNN robusti ottenuti utilizzando una particolare 
architettura di rete neurale nota come Time Delay Neural Network (TDNN).
Le TDNNs sono particolarmente adatte per il trattamento di sequenze temporali come lo speech, in quanto possono catturare contesti temporali 
più ampi senza la necessità di un numero eccessivo di parametri, grazie all'uso di connessioni ritardate e strati con pooling temporale. 
L'architettura TDNN consente alla rete di apprendere relazioni a lungo termine tra le caratteristiche vocali a livello di segmenti. 
Nello specifico, gli x-vectors sono estratti da una DNN. 

