\chapter{Introduzione}
Nel contesto dell'analisi intelligente dei segnali un task molto importante e di notevole interesse scientifico-industriale 
è il riconoscimento del parlante o in genere dell'identificazione di chi parla a discapito di impostori. Il seguente progetto affronta la tematica
della \textbf{speaker identification} e della \textbf{speaker verification}, analizzando le principali tecniche che permettono di ottenere
una rappresentazione alternativa a partire dalle tracce audio al fine di realizzare il riconoscimento degli speakers. 
Verranno quindi appprofondite le tecniche utilizzate e le \textbf{features} che entrerannno in gioco, analizzando come e quali sistemi permettono l'analisi di queste
caratteristiche. Infine verrà condotta una fase sperimentale dove verranno confrontate diverse tecniche e metodologie proposte in ambito di ricerca. \\
Il seguente progetto è strutturato nel seguente modo:
\begin{itemize}
    \item Il capitolo \ref{ch:speakerID} prenderà in considerazione il problema, analizzando cosa viene fatto in letteratura scientifica
    \item Il capitolo \ref{ch:features} analizzerà quali sono le features largamente utilizzate per poter effettuare speaker identification e verification 
    \item Il capitolo \ref{ch:speakermodels} fornirà una overview dei sistemi utilizzati per analizzare le features descritte prima, entrando anche nel dettaglio dell'architettura utilizzata
        per realizzare speaker identification e verification in un unica pipeline integrata
    \item Il capitolo \ref{ch:esperimenti} infine valuterà gli esperimenti condotti, descrivendo dataset e risultati ottenuti
\end{itemize}

% Nella introduzione parlare delle idee del progetto in generale e in quale aspetto ci siamo
% concentrati e perché, parlando anche di possibili scenari applicativi e motivazioni.
% Descrivere nella pratica (prendendo spunto dai vari paper scientifici) 
% di cosa tratteremo:
% \begin{itemize}
%     \item Capitolo 1: speaker identification e Verification, pipeline e come fare il tutto
%     \item Capitolo 2: speech featues, con particolare appiglio su MFCC, Mel Filterbanks, I-Vector, X-Vector
%     \item Capitolo 3: Speaker Model e Backend, introduzione alle tecniche utilizzate finora, dicendo quelle che analizzeremo
%     \item Capitolo 4: Sezione sperimentale 
%     \item Capitolo 4.1: descrizione del dataset utilizzato e della pipeline di lavoro, analizzando quali speaker stiamo utilizzando (in termini di Spettrogrammi e Classificazione maschio femmina)
%     \item Capitolo 4.1: utilizzo degli MFCC e descrizione dei risultati sperimentali sia per la identification che la verification con EER
%     \item Capitolo 4.2: utilizzo degli I-Vector e architettura della letteratura
%     \item Capitolo 4.3: utilizzo degli x-vector e arcitettura della letteratura per gli embeddings 
% \end{itemize}
