\chapter{Introduzione}
Nel contesto dell'analisi intelligente dei segnali un task molto importante e di notevole interesse scientifico-industriale 
è il riconoscimento del parlante o \textbf{speaker verification} o in genere dell'identificazione di chi parla o è autorizzato a parlare rispetto a chi non lo sia, \textbf{speaker identification}. 
Il seguente progetto affronta la tematica della \textbf{speaker identification} e della \textbf{speaker verification}, analizzando le principali tecniche che permettono di ottenere
una rappresentazione alternativa a partire dalle tracce audio al fine di realizzare il riconoscimento degli speakers. 
Verranno quindi appprofondite le tecniche utilizzate dalla letteratura scientifica e le \textbf{features} che entrano in gioco, analizzando come e quali sistemi permettono l'analisi di queste
caratteristiche. Infine verrà condotta una fase sperimentale dove verranno confrontate diverse tecniche e metodologie proposte in ambito di ricerca. \\
Il lavoro è strutturato nel seguente modo:
\begin{itemize}
    \item Il capitolo \ref{ch:speakerID} prenderà in considerazione il problema sotto un punto di vista formale, analizzando cosa viene fatto in letteratura scientifica
    \item Il capitolo \ref{ch:features} analizzerà invece quali sono le features largamente utilizzate per poter effettuare \textit{speaker identification e verification }
    \item Il capitolo \ref{ch:speakermodels} fornirà una overview dei sistemi utilizzati per analizzare le features descritte prima, entrando anche nel dettaglio dell'architettura proposta ed utilizzata
        nella sezione sperimentale per realizzare speaker identification e verification in un unica pipeline integrata
    \item Il capitolo \ref{ch:esperimenti} infine valuterà gli esperimenti condotti, descrivendo dataset e risultati ottenuti, oltre che fornire le conclusioni finali del lavoro
\end{itemize}